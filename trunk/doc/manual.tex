\documentclass[a4paper,11pt]{article}
\usepackage[pdftex,usenames,dvipsnames]{color}
\usepackage[spanish]{babel}
\usepackage[utf8]{inputenc}
\usepackage{amsmath, amsfonts, amssymb}
\usepackage[pdftex]{color,graphicx}
\title{\textbf{Kooter: Manual de Usuario}}
\author{by \textit{Kooter Inc.}}
\date{\today}

\begin{document}
\maketitle
\newpage
\tableofcontents
\newpage
\section{Introducción}
Este manual tiene como objetivo enseñar al usuario de \emph{Kooter} como se utiliza para lograr explotarlo al máximo. Primero se explicarán que comandos se encuentran disponibles y luego como lograr un bueno manejo del mismo.

\section{Kooter}
\subsection{Comandos básicos}
\begin{itemize}
	\item \textbf{``clear":} Limpia la pantalla.
	\item \textbf{``activaSp":} Activa el salva pantallas (\emph{screensaver}).
	\item \textbf{``echo \emph{string}":} Imprime el \emph{string} en pantalla.
	\item \textbf{``garbage":} Imprime basura en la pantalla.
	\item \textbf{``dispImg":} Muestra una imagen en pantalla.
	\item \textbf{``uname":} Muestra el nombre y la versión del Sistema operativo.
	\item \textbf{``pwd":} Muestra el directorio actual de trabajo (simulado).
	\item \textbf{``ls":} Muestra archivos y carpetas en el directorio actual (simulado).
	\item \textbf{``help":} Muestra en pantalla los comandos que se encuentran disponibles.
	
	
\end{itemize}
\subsection{Comandos avanzados}
\begin{itemize}
	\item \textbf{``setTimeSp \emph{time}":} Configura el tiempo que tarda el salva pantallas en activarse, para que tarde \emph{time} segundos. 
	\item \textbf{``mario":} Inicia un videojuego.
\end{itemize}

\subsection{Periféricos}
\subsubsection{Dispositivos soportados}
\emph{Kooter} soporta los siguientes dispositivos:
\begin{itemize}
	\item \textbf{``Mouse PS/2"} 
	\item \textbf{``Mouse USB"} 
	\item \textbf{``Teclado de idioma inglés"}
\end{itemize}
\subsubsection{Funcionalidad de los dispositivos soportados}
\begin{enumerate}
	\item \begin{large}\texttt{Mouse:} \end{large}
\emph{Kooter} ofrece importantes funciones para los periféricos soportados. Primero, para el caso del mouse, tanto el PS/2 como el USB, existen las funciones de copiar y pegar. Con el botón izquierdo del mouse se selecciona la posición inicial que se quiere copiar y se mantiene apretado hasta donde se quiera copiar. Esto determinará un rectángulo que es lo que se copiará. Luego para poder utilizar la función de pegar, simplemente se debe hacer click derecho del mouse y se pegará lo copiado en la siguiente posición a escribir.
\\
\\
\fbox{\textbf{Nota:} El área copiada se pegará tal cual fue se hallaba en pantalla.}
\\

	\item \begin{large}\texttt{Teclado:} \end{large}
El teclado se halla en idioma inglés, cualquier carácter que no pertenezca a este lenguaje no será soportado por el teclado. Es decir, \emph{Kooter} no garantiza cualquier otro carácter que no pertenezca al inglés. Los carácteres soportados son las letras, los números, el shift, los símbolos de puntuación, el \emph{backspace} y la barra espaciadora (\emph{spacebar}).
\end{enumerate}

\subsubsection{Forma de ejecutar \emph{Kooter}}
\begin{enumerate}
	\item Colocar el CD diskette o pendrive booteable.
	\item Encender la PC y hacerla bootear por este medio.
	\item Si esto se hizo de manera correcta se podrá ver una pantalla para elegir el Sistema operativo o Kernel a bootear.
	\item Se debe seleccionar \emph{Kooter}.
	\item Una vez hecho esto, luego de la pantalla de presentación ya se podrá proceder a ingresar comandos en la línea de comando.
\end{enumerate}

\subsubsection{Versión}
Al día de impresión de este manual, \emph{Kooter} se encuentra en su versión 1.0.

\newpage

\section{Contacto}
Por cualquier inconveniente o duda puede contactarse a las siguientes direcciones de e-mail:
\begin{description}
	\item[Manuel Aráoz:]  \texttt{maraoz@alu.itba.edu.ar}
	\item[Pablo Giorgi:]  \texttt{pgiorgi@alu.itba.edu.ar}
	\item[Matías Williams:]  \texttt{mwilliam@alu.itba.edu.ar}
\end{description}




\end{document}
