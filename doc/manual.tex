\documentclass[a4paper,11pt]{article}
\usepackage[pdftex,usenames,dvipsnames]{color}
\usepackage[spanish]{babel}
\usepackage[utf8]{inputenc}
\usepackage{amsmath, amsfonts, amssymb}
\usepackage[pdftex]{color,graphicx}
\title{Kooter: Manual de Usuario}
\author{by Kooter Co.}
\date{\today}

\begin{document}
\maketitle
\newpage
\tableofcontents
\newpage
\section{Introducción}
Este manual tiene como objetivo enseñar al usuario de \emph{Kooter} como se utiliza para lograr explotarlo al máximo. Primero se explicarán que comandos se encuentran disponibles y luego como lograr un bueno manejo del mismo.

\section{Kooter}
\subsection{Comandos básicos}
\begin{itemize}
	\item \textbf{``clear":} Limpia la pantalla.
	\item \textbf{``activaSp":} Activa el salva pantallas (\emph{screensaver}).
	\item \textbf{``echo \emph{string}":} Imprime el \emph{string} en pantalla.
	\item \textbf{``garbage":} Imprime basura en la pantalla.
	\item \textbf{``dispImg":} Muestra una imagen en pantalla.
	
	
\end{itemize}
\subsection{Comandos avanzados}
\begin{itemize}
	\item \textbf{``setTimeSp \emph{time}":} Configura el tiempo que tarda el salva pantallas en activarse, para que tarde \emph{time} segundos. 
	\item \textbf{``mario":} Inicia un videojuego.
\end{itemize}

\subsection{Periféricos}
\subsubsection{Dispositivos soportados}
\emph{Kooter} soporta los siguientes dispositivos:
\begin{itemize}
	\item \textbf{``Mouse PS/2"} 
	\item \textbf{``Mouse USB"} 
	\item \textbf{``Teclado de idioma inglés"}
\end{itemize}
\subsubsection{Funcionalidad de los dispositivos soportados}
\begin{enumerate}
	\item \begin{large}\texttt{Mouse:} \end{large}
\emph{Kooter} ofrece importantes funciones para los periféricos soportados. Primero, para el caso del mouse, tanto el PS/2 como el USB, existen las funciones de copiar y pegar. Con el botón izquierdo del mouse se selecciona la posición inicial que se quiere copiar y se mantiene apretado hasta donde se quiera copiar. Esto determinará un rectángulo que es lo que se copiará. Luego para poder utilizar la función de pegar, simplemente se debe hacer click derecho del mouse y se pegará lo copiado en la siguiente posición a escribir.
\\
\\
\fbox{\textbf{Nota:} El área copiada se pegará tal cual fue se hallaba en pantalla.}
\\

	\item \begin{large}\texttt{Teclado:} \end{large}
El teclado se halla en idioma inglés, cualquier carácter que no pertenezca a este lenguaje no será soportado por el teclado. Es decir, \emph{Kooter} no garantiza cualquier otro carácter que no pertenezca al inglés. Los carácteres soportados son las letras, los números, el shift, los símbolos de puntuación, el \emph{backspace} y la barra espaciadora (\emph{spacebar}).
\end{enumerate}



\end{document}
